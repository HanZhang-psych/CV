%!TEX TS-program = xelatex
%!TEX encoding = UTF-8 Unicode
% Awesome CV LaTeX Template for CV/Resume
%
% This template has been downloaded from:
% https://github.com/posquit0/Awesome-CV
%
% Author:
% Claud D. Park <posquit0.bj@gmail.com>
% http://www.posquit0.com
%
%
% Adapted to be an Rmarkdown template by Mitchell O'Hara-Wild
% 23 November 2018
%
% Template license:
% CC BY-SA 4.0 (https://creativecommons.org/licenses/by-sa/4.0/)
%
%-------------------------------------------------------------------------------
% CONFIGURATIONS
%-------------------------------------------------------------------------------
% A4 paper size by default, use 'letterpaper' for US letter
\documentclass[11pt,a4paper,]{awesome-cv}

% Configure page margins with geometry
\usepackage{geometry}
\geometry{left=1.4cm, top=.8cm, right=1.4cm, bottom=1.8cm, footskip=.5cm}


% Specify the location of the included fonts
\fontdir[fonts/]

% Color for highlights
% Awesome Colors: awesome-emerald, awesome-skyblue, awesome-red, awesome-pink, awesome-orange
%                 awesome-nephritis, awesome-concrete, awesome-darknight

\definecolor{awesome}{HTML}{414141}

% Colors for text
% Uncomment if you would like to specify your own color
% \definecolor{darktext}{HTML}{414141}
% \definecolor{text}{HTML}{333333}
% \definecolor{graytext}{HTML}{5D5D5D}
% \definecolor{lighttext}{HTML}{999999}

% Set false if you don't want to highlight section with awesome color
\setbool{acvSectionColorHighlight}{true}

% If you would like to change the social information separator from a pipe (|) to something else
\renewcommand{\acvHeaderSocialSep}{\quad\textbar\quad}

\def\endfirstpage{\newpage}

%-------------------------------------------------------------------------------
%	PERSONAL INFORMATION
%	Comment any of the lines below if they are not required
%-------------------------------------------------------------------------------
% Available options: circle|rectangle,edge/noedge,left/right

\name{Han}{Zhang}

\position{Postdoctoral Research Fellow}
\address{Department of Psychology, University of Michigan}

\email{\href{mailto:hanzh@umich.edu}{\nolinkurl{hanzh@umich.edu}}}
\homepage{sites.lsa.umich.edu/hanzh}
\googlescholar{ujW-lXkAAAAJ}
\researchgate{Han\_Zhang175}
\github{HanZhang-psych}
\osf{nrtbk}
\twitter{\_HanZhang\_}

% \gitlab{gitlab-id}
% \stackoverflow{SO-id}{SO-name}
% \skype{skype-id}
% \reddit{reddit-id}


\usepackage{booktabs}

\providecommand{\tightlist}{%
	\setlength{\itemsep}{0pt}\setlength{\parskip}{0pt}}

%------------------------------------------------------------------------------



% Pandoc CSL macros
\newlength{\cslhangindent}
\setlength{\cslhangindent}{1.5em}
\newlength{\csllabelwidth}
\setlength{\csllabelwidth}{2em}
\newenvironment{CSLReferences}[2] % #1 hanging-ident, #2 entry spacing
 {% don't indent paragraphs
  \setlength{\parindent}{0pt}
  % turn on hanging indent if param 1 is 1
  \ifodd #1 \everypar{\setlength{\hangindent}{\cslhangindent}}\ignorespaces\fi
  % set entry spacing
  \ifnum #2 > 0
  \setlength{\parskip}{#2\baselineskip}
  \fi
 }%
 {}
\usepackage{calc}
\newcommand{\CSLBlock}[1]{#1\hfill\break}
\newcommand{\CSLLeftMargin}[1]{\parbox[t]{\csllabelwidth}{\honortitlestyle{#1}}}
\newcommand{\CSLRightInline}[1]{\parbox[t]{\linewidth - \csllabelwidth}{\honordatestyle{#1}}}
\newcommand{\CSLIndent}[1]{\hspace{\cslhangindent}#1}

\begin{document}

% Print the header with above personal informations
% Give optional argument to change alignment(C: center, L: left, R: right)
\makecvheader

% Print the footer with 3 arguments(<left>, <center>, <right>)
% Leave any of these blank if they are not needed
% 2019-02-14 Chris Umphlett - add flexibility to the document name in footer, rather than have it be static Curriculum Vitae
\makecvfooter
  {June 2023}
    {Han Zhang~~~·~~~Curriculum Vitae}
  {\thepage}


%-------------------------------------------------------------------------------
%	CV/RESUME CONTENT
%	Each section is imported separately, open each file in turn to modify content
%------------------------------------------------------------------------------



\hypertarget{research-interests}{%
\section{Research Interests}\label{research-interests}}

Attention; distraction; mind-wandering; eye movements

\hypertarget{education}{%
\section{Education}\label{education}}

\begin{cventries}
    \cventry{Postdoctoral Research Fellow, Department of Psychology}{University of Michigan}{Ann Arbor, MI, USA}{06/2020-present}{\begin{cvitems}
\item Advisor: Dr. John Jonides
\end{cvitems}}
    \cventry{Doctor of Philosophy, Education and Psychology}{University of Michigan}{Ann Arbor, MI, USA}{09/2014-05/2020}{\begin{cvitems}
\item Dissertation: Mind-wandering: What Can We Learn from Eye Movements?
\end{cvitems}}
    \cventry{Master of Science, Psychology}{University of Michigan}{Ann Arbor, MI, USA}{09/2014-04/2018}{}\vspace{-4.0mm}
    \cventry{Bachelor of Science, Psychology}{Beijing Normal University}{Beijing, China}{09/2010-07/2014}{}\vspace{-4.0mm}
\end{cventries}

\hypertarget{awards-and-grants}{%
\section{Awards and Grants}\label{awards-and-grants}}

\hypertarget{since-postdoc}{%
\subsection{Since Postdoc}\label{since-postdoc}}

\begin{cventries}
    \cventry{Standard Grant, National Science Foundation}{Probing attentional allocation with a novel forced-response method}{2023}{\$659,877}{\begin{cvitems}
\item Role: Key Personnel (Postdoc)
\end{cvitems}}
    \cventry{R21, National Institute of Mental Health}{Investigating interference-control in ADHD using a novel forced-response method}{2023}{\$429,000}{\begin{cvitems}
\item Role: Key Personnel (Postdoc)
\end{cvitems}}
\end{cventries}

\hypertarget{before-postdoc}{%
\subsection{Before Postdoc}\label{before-postdoc}}

\begin{cventries}
    \cventry{School of Education, University of Michigan}{Stanley E. and Ruth B. Dimond Best Dissertation Award}{2020}{\$500}{\begin{cvitems}
\item Dissertation: Mind-wandering: What Can We Learn from Eye Movements?
\end{cvitems}}
    \cventry{School of Education, University of Michigan}{2020 ProQuest Distinguished Dissertation Awards (Nomination)}{2020}{}{}\vspace{-4.0mm}
    \cventry{Rackham Graduate School, University of Michigan}{Rackham One-Term Dissertation Fellowship}{2019}{\$17417}{}\vspace{-4.0mm}
    \cventry{Rackham Graduate School, University of Michigan}{Rackham Graduate Student Research Grant}{2019}{\$3000}{}\vspace{-4.0mm}
    \cventry{Rackham Graduate School, University of Michigan}{Rackham Conference Travel Grant}{2019, 2018, 2017}{\$1050}{}\vspace{-4.0mm}
    \cventry{Rackham Graduate School, University of Michigan}{Rackham Summer Training Award}{2018}{\$3700}{}\vspace{-4.0mm}
    \cventry{School of Education, University of Michigan}{Professor \& Mrs. Cho-Yee To Travel Grant}{2017, 2015}{\$500}{}\vspace{-4.0mm}
    \cventry{Ministry of Education of the People's Republic of China}{China National Scholarship}{2014}{\$1500}{}\vspace{-4.0mm}
    \cventry{Faculty of Psychology, Beijing Normal University}{The First Rank Academic Scholarship}{2014, 2013}{\$300}{}\vspace{-4.0mm}
    \cventry{Beijing Municipal Commission of Education}{Beijing Student Research and Entrepreneurial Action Plan}{2013}{\$1500}{}\vspace{-4.0mm}
    \cventry{Faculty of Psychology, Beijing Normal University}{Undergraduate Student Research Grant}{2012}{\$280}{}\vspace{-4.0mm}
\end{cventries}

\hypertarget{peer-reviewed-articles}{%
\section{Peer-reviewed Articles}\label{peer-reviewed-articles}}

\hypertarget{bibliography}{}
\leavevmode\vadjust pre{\hypertarget{ref-zhang_scene_2021}{}}%
\textbf{Zhang, H.}, Anderson, N. C., \& Miller, K. F. (2021). Scene
meaningfulness guides eye movements even during mind-wandering.
\emph{Attention, Perception, \& Psychophysics}.
\url{https://doi.org/10.3758/s13414-021-02370-6}

\leavevmode\vadjust pre{\hypertarget{ref-zhang_malleability_2021}{}}%
\textbf{Zhang, H.}, Abagis, T. R., \& Jonides, J. (2021). The
malleability of attentional capture. \emph{Visual Cognition}.
\url{https://doi.org/10.1080/13506285.2021.1915903}

\leavevmode\vadjust pre{\hypertarget{ref-zhang_pre-trial_2021}{}}%
\textbf{Zhang, H.}, \& Jonides, J. (2021). Pre-trial gaze stability
predicts momentary slips of attention. \emph{EMICS '21}. ACM CHI '21
workshop on eye movements as an interface to cognitive state.
\url{https://psyarxiv.com/bv2uc/}

\leavevmode\vadjust pre{\hypertarget{ref-zhang_refixation_2020}{}}%
\textbf{Zhang, H.}, Anderson, N. C., \& Miller, K. F. (2020). Refixation
patterns of mind-wandering during real-world scene perception.
\emph{Journal of Experimental Psychology: Human Perception and
Performance}, \emph{47}(1), 36--52.
\url{https://doi.org/10.1037/xhp0000877}

\leavevmode\vadjust pre{\hypertarget{ref-zhang_wandering_2020}{}}%
\textbf{Zhang, H.}, Miller, K. F., Sun, X., \& Cortina, K. S. (2020).
Wandering eyes: Eye movements during mind wandering in video lectures.
\emph{Applied Cognitive Psychology}, \emph{34}(2), 449--464.
\url{https://doi.org/10.1002/acp.3632}

\leavevmode\vadjust pre{\hypertarget{ref-zhang_missing_2020}{}}%
\textbf{Zhang, H.}, Qu, C., Miller, K. F., \& Cortina, K. S. (2020).
Missing the joke: Reduced rereading of garden-path jokes during
mind-wandering. \emph{Journal of Experimental Psychology: Learning,
Memory, and Cognition}, \emph{46}(4), 638--648.
\url{https://doi.org/10.1037/xlm0000745}

\leavevmode\vadjust pre{\hypertarget{ref-zhang_how_2018}{}}%
\textbf{Zhang, H.}, Miller, K. F., Cleveland, R., \& Cortina, K. S.
(2018). How listening to music affects reading: Evidence from eye
tracking. \emph{Journal of Experimental Psychology: Learning, Memory,
and Cognition}, \emph{44}(11), 1778--1791.
\url{https://doi.org/10.1037/xlm0000544}

\hypertarget{book-chapters}{%
\section{Book Chapters}\label{book-chapters}}

\hypertarget{bibliography}{}
\leavevmode\vadjust pre{\hypertarget{ref-michal_cognitive_2021}{}}%
Michal, A., Fansher, M., Xin, S., \textbf{Zhang, H.}, \& Shah, P.
(2021). Cognitive development: Methods and illustrative examples. In
\emph{The oxford handbook of educational psychology}. Oxford University
Press.

\hypertarget{under-review}{%
\section{Under Review}\label{under-review}}

\hypertarget{bibliography}{}
\leavevmode\vadjust pre{\hypertarget{ref-adkins_what_2023}{}}%
Adkins, T. J., \textbf{Zhang, H.}, Jonides, J., \& Lee, T. G. (2023).
\emph{What happens after an error?} BioArXiv.
\url{https://doi.org/10.1101/2022.03.17.484792}

\leavevmode\vadjust pre{\hypertarget{ref-lee_forced-response_2023}{}}%
Lee, T. G., Sellers, J., Jonides, J., \& \textbf{Zhang, H.} (2023).
\emph{The forced-response method: A new chronometric approach to measure
conflict processing}. PsyArXiv.
\url{https://doi.org/10.31234/osf.io/byzqf}

\leavevmode\vadjust pre{\hypertarget{ref-osborne_association_2023}{}}%
Osborne, J., \textbf{Zhang, H.}, Carlson, M., Shah, P., \& Jonides, J.
(2023). \emph{The association between different sources of distraction
and symptoms of attention deficit hyperactivity disorder.}

\leavevmode\vadjust pre{\hypertarget{ref-zhang_lingering_2023}{}}%
\textbf{Zhang, H.}, Abagis, T. R., Steeby, C. J., \& Jonides, J. (2023).
\emph{Lingering on distraction: Examining distractor rejection in adults
with ADHD}. PsyArXiv. \url{https://doi.org/10.31234/osf.io/cqe34}

\leavevmode\vadjust pre{\hypertarget{ref-zhang_does_2023}{}}%
\textbf{Zhang, H.}, Miller, K. F., \& Jonides, J. (2023). \emph{Does
mind-wandering affect distractor suppression?} PsyArXiv.
\url{https://doi.org/10.31234/osf.io/2h6kw}

\leavevmode\vadjust pre{\hypertarget{ref-zhang_d_2023}{}}%
\textbf{Zhang, H.}, Miyake, A., Shah, P., \& Jonides, J. (2023). \emph{A
d factor? Understanding individual differences in distractibility}.
PsyArXiv. \url{https://doi.org/10.31234/osf.io/5nzs2}

\leavevmode\vadjust pre{\hypertarget{ref-zhang_association_2023}{}}%
\textbf{Zhang, H.}, Xin, S., \& Miyake, A. (2023). \emph{On the
association between the self-control scale and social desirability
measures: A meta-analysis and systematic review}. PsyArXiv.
\url{https://doi.org/10.31234/osf.io/j4vy8}

\hypertarget{selected-presentations}{%
\section{Selected Presentations}\label{selected-presentations}}

\hypertarget{bibliography}{}
\leavevmode\vadjust pre{\hypertarget{ref-zhang_pm_2022}{}}%
\textbf{Zhang, H.}, Adkins, T., Lee, T., \& Jonides, J. (2022,
November). \emph{How does the priority map change over time?} Oral
presentation at the psychonomic society's 63nd annual meeting.

\leavevmode\vadjust pre{\hypertarget{ref-zhang_uncovering_2021}{}}%
\textbf{Zhang, H.}, Miyake, A., Shah, P., \& Jonides, J. (2021,
November). \emph{Uncovering the structure of individual differences in
distractibility}. Oral presentation at the psychonomic society's 62nd
annual meeting.

\leavevmode\vadjust pre{\hypertarget{ref-zhang_mind-wandering_2020}{}}%
\textbf{Zhang, H.}, Anderson, N. C., \& Miller, K. F. (2020, November).
\emph{Mind-wandering during scene perception: On the role of meaning and
salience}. Poster presented at the psychonomic society's 61st annual
meeting.

\leavevmode\vadjust pre{\hypertarget{ref-zhang_scan-paths_2020}{}}%
\textbf{Zhang, H.}, Anderson, N. C., \& Miller, K. F. (2020, April).
\emph{Scan-paths of mind-wandering during real-world scene perception}.
The 92nd annual meeting of the midwestern psychological association
(conference canceled due to covid-19).

\leavevmode\vadjust pre{\hypertarget{ref-zhang_reduced_2019}{}}%
\textbf{Zhang, H.}, Qu, C., Miller, K. F., \& Cortina, K. S. (2019,
November). \emph{Reduced re-reading of garden-path jokes during mindless
reading}. Poster presented at the psychonomic society's 60th annual
meeting.

\leavevmode\vadjust pre{\hypertarget{ref-zhang_mind-wandering_2019}{}}%
\textbf{Zhang, H.}, Miller, K. F., Cortina, K. S., \& Jiang, T. (2019,
June). \emph{Mind-wandering in college classrooms: A mobile eye-tracking
study}. Poster presented at SARMAC XIII.

\leavevmode\vadjust pre{\hypertarget{ref-fischer_examining_2019}{}}%
Fischer, A., \& \textbf{Zhang, H.} (2019, May). \emph{Examining the
effect of word predictability during mindless reading}. Poster presented
at the association for psychological science annual meeting.

\leavevmode\vadjust pre{\hypertarget{ref-zhang_what_2019}{}}%
\textbf{Zhang, H.}, \& Shah, P. (2019, May). \emph{What can iPhone's
screen time tell about your cognitive functioning?} Poster presented at
the association for psychological science annual meeting.

\leavevmode\vadjust pre{\hypertarget{ref-zhang_mind-wandering_2019-1}{}}%
\textbf{Zhang, H.} (2019, April). \emph{Mind-wandering: What can we
learn from eye-movements?} Oral presentation at the combined program in
education and psychology brownbag.

\leavevmode\vadjust pre{\hypertarget{ref-zhang_scan-paths_2019}{}}%
\textbf{Zhang, H.}, Miller, K. F., \& Sun, X. (2019, March).
\emph{Scan-paths of mind-wandering during video lectures}. Poster
presented at the international convention of psychological science.

\leavevmode\vadjust pre{\hypertarget{ref-zhang_how_2018}{}}%
\textbf{Zhang, H.}, \& Miller, K. F. (2018, November). \emph{How
irrelevant speech affects reading: The role of word predictability}.
Poster presented at the psychonomic society's 59th annual meeting.

\leavevmode\vadjust pre{\hypertarget{ref-qu_all_2018}{}}%
Qu, C., \& \textbf{Zhang, H.} (2018, May). \emph{All joking aside:
Mind-wandering impairs processing of garden-path jokes}. Poster
presented at the association for psychological science annual meeting.

\leavevmode\vadjust pre{\hypertarget{ref-zhang_wandering_2018}{}}%
\textbf{Zhang, H.}, Miller, K. F., \& Sun, X. (2018, May). \emph{The
wandering eyes: Mind-wandering during video lectures is associated with
oculomotor behaviors}. Poster presented at the association for
psychological science annual meeting.

\leavevmode\vadjust pre{\hypertarget{ref-zhang_how_2016}{}}%
\textbf{Zhang, H.}, Miller, K. F., Cleveland, R., \& Cortina, K. S.
(2016, May). \emph{How listening to music affects reading: Evidence from
eye tracking}. Poster presented at the association for psychological
science annual meeting.

\hypertarget{teaching-experience}{%
\section{Teaching Experience}\label{teaching-experience}}

\hypertarget{primary-instructor}{%
\subsection{Primary Instructor}\label{primary-instructor}}

\begin{cventries}
    \cventry{Course design; weekly lectures; grading assignments and exams; office hours}{EDUC 391: Educational Psychology and Human Development}{09/2017 - 12/2017}{}{}\vspace{-4.0mm}
\end{cventries}

\hypertarget{graduate-student-instructor}{%
\subsection{Graduate Student
Instructor}\label{graduate-student-instructor}}

\begin{cventries}
    \cventry{Grading student essays and exams; office hours; supervising student research projects}{PSYCH 457 : Research in Educational and Cross-Cultural Settings}{01/2020 - 04/2020}{}{}\vspace{-4.0mm}
    \cventry{Grading student essays and exams; office hours; supervising student research projects}{PSYCH 457 : Research in Educational and Cross-Cultural Settings}{01/2019 - 04/2019}{}{}\vspace{-4.0mm}
    \cventry{Three review sessions per week; grading assignments and exams; office hours}{PSYCH 111: Introduction to Psychology}{01/2018 - 04/2018}{}{}\vspace{-4.0mm}
    \cventry{Grading student essays and exams; office hours; supervising student research projects}{PSYCH 457 : Research in Educational and Cross-Cultural Settings}{01/2017 - 04/2017}{}{}\vspace{-4.0mm}
    \cventry{Three review sessions per week; grading assignments and exams; office hours}{PSYCH 240: Intro to Cognitive Psychology}{09/2016 - 12/2016}{}{}\vspace{-4.0mm}
    \cventry{Grading student essays and exams; office hours; supervising student research projects}{PSYCH 457 : Research in Educational and Cross-Cultural Settings}{01/2016 - 04/2016}{}{}\vspace{-4.0mm}
\end{cventries}

\hypertarget{program-assistant}{%
\subsection{Program Assistant}\label{program-assistant}}

\begin{cventries}
    \cventry{Supervising joint research projects between US and Chinese students; organizing student activities}{Global Course Connection: Beijing Normal University}{2015 - 2019}{}{}\vspace{-4.0mm}
\end{cventries}

\hypertarget{grant-writing-experience}{%
\section{Grant-writing Experience}\label{grant-writing-experience}}

\begin{cventries}
    \cventry{R21, National Institute of Mental Health}{Investigating interference-control in ADHD using a novel forced-response method}{2022}{Funded}{\begin{cvitems}
\item Role: Conceptualiation; original draft preparation; formal analysis; reviewing \& editing
\end{cvitems}}
    \cventry{National Science Foundation}{Probing attentional allocation with a novel forced-response method}{2022}{Funded}{\begin{cvitems}
\item Role: Conceptualiation; original draft preparation; formal analysis; reviewing \& editing
\end{cvitems}}
    \cventry{Merck Investigator Studies Program}{Testing the efficacy of interactive decision aids on vaccine attitudes and uptake using a Bayesian precision estimates}{2022}{Pending}{\begin{cvitems}
\item Role: reviewing \& editing
\end{cvitems}}
    \cventry{National Science Foundation}{Uncovering mechanisms of interference-resolution with a novel forced-response method}{2022}{Not Funded}{\begin{cvitems}
\item Role: Conceptualiation; original draft preparation; formal analysis; reviewing \& editing
\end{cvitems}}
    \cventry{National Science Foundation}{Dissecting Distractibility by the Capture of Attention}{2021}{Not Funded}{\begin{cvitems}
\item Role: Original draft preparation; formal analysis; reviewing \& editing
\end{cvitems}}
    \cventry{National Science Foundation}{Mitigating distraction from external and internal sources}{2020}{Not Funded}{\begin{cvitems}
\item Role: Original draft preparation; formal analysis; reviewing \& editing
\end{cvitems}}
    \cventry{R01, National Institute of Mental Health}{Failure to resist external and internal distraction in ADHD}{2019}{Not Funded}{\begin{cvitems}
\item Role: Formal analysis; original draft preparation; reviewing \& editing
\end{cvitems}}
\end{cventries}

\hypertarget{professional-service}{%
\section{Professional Service}\label{professional-service}}

\hypertarget{ad-hoc-reviewer}{%
\subsection{Ad Hoc Reviewer}\label{ad-hoc-reviewer}}

\begin{itemize}
\tightlist
\item
  Memory \& Cognition
\item
  NeuroImage
\item
  Visual Cognition
\item
  Cognitive Science
\item
  Scientific Studies of Reading
\item
  Frontiers in Psychiatry
\item
  Psychological Research
\item
  Cognition
\item
  Scientific Reports
\end{itemize}

\hypertarget{references}{%
\section{References}\label{references}}

\begin{cventries}
    \cventry{Edward E. Smith Professor of Psychology and Neuroscience; Co-Director, functional MRI Center, University of Michigan; Senior Editor, Psychological Science}{John Jonides, Ph.D.}{jjonides@umich.edu; 734-764-0192}{}{}\vspace{-4.0mm}
    \cventry{Professor of Psychology and Education,  University of Michigan}{Kevin F. Miller, Ph.D.}{kevinmil@umich.edu; 734-615-1800}{}{}\vspace{-4.0mm}
    \cventry{Professor of Cognition \& Cognitive Neuroscience and Educational Psychology, University of Michigan}{Priti Shah, Ph.D.}{priti@umich.edu; 734-615-3745}{}{}\vspace{-4.0mm}
    \cventry{Professor of Psychology, University of Michigan}{Kai S. Cortina, Ph.D.}{schnabel@umich.edu; 734-615-3809}{}{}\vspace{-4.0mm}
\end{cventries}



\end{document}
